%!TEX root = ../thesis.tex

\chapter{Abschlussbetrachtung und Ausblick} % (fold)
\label{cha:zusammenfassung_und_ausblick}

Zum Abschluss dieser Masterarbeit, soll noch einmal kurz alle vorherigen Kapitel mit dem wichtigsten Inhalten zusammengefasst werden. Danach folgt eine kritische Würdigung in der diese Arbeit rückblickend bewertet wird und außerdem ein Ausblick auf weiterführende Themen. 

\section{Zusammenfassung} % (fold)
\label{sec:zusammenfassung} 

Diskussionen sind ein wichtiger Bestandteil des E-Learnings, aber oftmals finden diese auf verschiedenen Plattformen verteilt statt. Dadurch verpassen Personen, die nur auf einer dieser Plattformen präsent sind, vielleicht für sie wichtige Informationen auf einer anderen. Außerdem werden die selben Diskussionsthemen immer wieder an unterschiedlichen Orten an den unterschiedlichsten Orten doppelt und dreifach behandelt. Aus diesem Grund sollte ein System entwickelt werden das den Austausch von Diskussionen zwischen den unterschiedlichen Plattformen ermöglicht. 

In Kapitel \ref{cha:analyse} wurde analysiert, welche Schritte für eine solche Synchronisation nötig sind. Zuerst musste eine Zwischenformat gefunden werden in das die Daten der unterschiedlichen Plattformen konvertiert werden konnten, da dieser Weg effizienter ist als die einzelnen Formate untereinander zu konvertieren. Als ein solches Zwischenformat bat sich die SIOC Ontologie in Verbindung mit FOAF wunderbar an. Ontologien bietet nicht nur einen guten Ansatz für die Integration von unterschiedlichen Datenformaten, mit RDF ist es sogar möglich dass Programme diese Daten verstehen und aus ihnen neues Wissen ableiten können. Um die Daten letztendlich in das Zwischenformat konvertieren und verarbeiten zu können, braucht es eine einheitliche Schnittstelle mit der dies umgesetzt werden kann. Zusätzlich wurde noch ein System gebraucht, dass die konvertierten Daten zwischen den Schnittstellen für die einzelnen Plattformen austauscht. Dazu eignete sich der Austausch über Nachrichten am besten, da dadurch die einzelnen Schnittstellen zeitlich und räumlich entkoppelt werden konnten und nicht voneinander abhängig sind. Als Basis für dieses Nachrichtensystem wurde Camel ausgewählt. Mit dieser Java-Bibliothek können die Routen, welche die Nachrichten transportieren, flexibel konfiguriert werden und sind so leicht erweiterbar. Privatsphäre spielt in der heutigen Zeit ebenfalls ein wichtige Rolle. Deswegen musste es Benutzern ermöglicht werden dem automatischen Lesen und/oder Schreiben für seine Beiträge zuzustimmen oder abzulehnen.

Das Kapitel \ref{cha:eigener_ansatz_social_online_community_connectors_socc_} widmet sich der Beschreibung des Systems, welches die Anforderungen aus Kapiel \ref{cha:analyse} erfüllen soll. Dieses SOCC genannte System definiert dazu Conectoren, welche die Schnittstellen zu den einzelnen Plattformen abstrahiert und die unterschiedlichen Formate in das SIOC-Format konvertiert. Ein Connector bestehen aus drei Komponenten die je für eine bestimmte Teilaufgabe des Systems verantwortlich sind. Der StructureReader hilft dabei die Struktur der jeweiligen Plattform zu lesen und in SIOC abzubilden, so dass die Beiträge von den richtigen Stellen gelesen und an die richtigen Stellen geschrieben werden können. Der PostReader ist dabei für das Lesen von Beiträgen und deren Konvertierung vom Format der gelesenen Plattform in das SIOC-Format zuständig. Für dem umgekehrten Weg wird dann der PostWriter eingesetzt. Für den Betrieb der Connectoren brauchen diese einige Informationen von den Benutzern, für den Zugriff auf die einzelnen APIs. Für diese Informationen konnte auf des vorhandene Service Modul von SIOC für API-Beschreibungen, die Basic Access Control Ontologie für die Autorisierung, sowie auf die aus vorhandenen Ontologien zusammengesetzte und erweiterte SIOC Services Authentication Ontologie zur Authentifizierung aufgebaut werden. Gespeichert wird alles in einen RDF-Triplestore und erlaubt so einen einfachen Zugriff auf die untereinander verbunden Datensätze. Für die Integration der Connectoren in Camel wurde die Komponente SOCC-Camel entwickelt. Mit den EIP aus Camel können so die Connectoren auf verschiedenste Weisen verbunden werden.

Im letzten Kapitel wurde die Implementierung von Connectoren beschrieben. Dabei wurde drauf eingegangen, auf welche Art und Weise man über die APIs der  Plattformen Moodle, Canvas, Facebook, Google+ und Youtube auf deren Daten zugreifen kann und welche Voraussetzungen dafür getroffen werden müssen. Außerdem wurde für jede Plattform gezeigt, wie ihre Struktur den Klassen von SIOC zugeordnet werden kann. Nicht bei allen Plattformen funktionierte alles Einwandfrei. Es wurde beschrieben welche Probleme auftraten und falls möglich eine Lösung präsentiert. Gleiches galt für die Implementierung von SOCC-Camel. Abgeschlossen wurde das Kapitel mit einem Proof of Concept in dem ein Schritt-für-Schritt-Beispiel für den Einsatz von SOCC und SOCC-Camel in einem Programm durchgearbeitet wurde.

% section zusammenfassung (end)

\section{kritische Würdigung} % (fold)
\label{sec:bewertung}

Abschließend betrachtet bin ich sehr zufrieden mit der Implementierung von meinen SOCC-Ansatz. Die Entscheidung für RDF und SIOC als Datenformat für die Integration von verteilen Diskussionen hat sich von Anfang an als eine gute erwiesen. Es stand zwar immer die Frage im Raum, ob sich in SIOC wirklich alle Inhalte der einzelnen Plattformen zusammenführen lassen, diese war aber vollkommen unbegründet. Bis das endgültige Konzept für SOCC feststand, waren aber trotzdem einige Versuche notwendig. Dies lag nicht nur daran, dass das Thema Datenintegration mit RDF und Ontologien für mich relativ neu war, aber auch dass mit fortschreitender Zeit man immer mehr Erfahrung sammelt und merkt was man am Anfang hätte besser machen können. Eine ebenfalls gute Wahl war auch die Verwendung von Camel für den Datenaustausch. Der Implementierungsaufwand war sehr gering und durch die Nutzung der beliebig konfigurierbaren Routen ist ein sehr flexibles System entstanden. 

Aber dass alles perfekt ist, kann man auch nicht sagen. Die in Abschnitt \ref{sub:allgemeine_probleme} beschrieben Lösungen zu doppelten Beiträgen und Rekonstruktion der Kommentarstruktur funktionieren zwar, sind aber alles andere als perfekt. Statt dem Einfügen eines Wasserzeichens könnte man nach einen Beitrag im Triplestore suchen, der dem vermeintlichen Duplikat gleicht. Die Rekonstruktion der Kommentarstruktur könnte auch besser funktionieren. Es ist schade, dass dies nur in eine Richtung reibungslos vonstatten geht. Um die Connectoren weiterhin unabhängig zu halten, müsste es in einer Verbesserung möglich sein, auch Daten über die Struktur der Plattformen auszutauschen. 

Auch sei noch folgendes zu RDF2Go erwähnt. Um es gleich zu sagen, RDF2Go ist ein guter Weg innerhalb von Java mit RDF-Daten zu arbeiten. Dadurch dass man nicht an einen bestimmten Triplestore und dessen API gebunden ist (dafür aber an die von RDF2Go) kann man für seine Zwecke den richtigen wählen. Auch der Generator zum Generieren von Java-Klassen aus Ontologien arbeitet besser als der von anderen Frameworks und das Arbeiten mir Klassen ist wesentlich einfacher als alle Triple einzeln zu manipulieren. Doch hat auch der Generator und die generierten Klassen so ihre Macken. Zum Beispiel werden für die einzelnen Eigenschaften keine Methoden generiert, um sich nur einen Wert einen Eigenschaft zurückgeben zu lassen. Man bekommt immer einen Iterator, aus dem das erste Ergebnis geholt werden musste. Diese fehlenden Methoden wurden aber dann von Hand hinzugefügt. Ein Problem mit der Generierung von Java-Klassen war das Fehlen der Mehrfachvererbung. Java kennt so etwas nicht, aber RDFSchema und OWL sehr wohl. So mussten einige Methoden aus der Java-Klasse \texttt{OnlineAccount} von FOAF in \texttt{UserAccount} von SIOC kopiert werden, um die Eigenschaften aus FOAF auch dort nutzen zu können.

% section bewertung (end)

\section{Ausblick} % (fold)
\label{sec:ausblick}

Einer der Vorteile der Trennung von der Datenkonvertierung mit SOCC und dem Datenaustausch mit SOCC-Camel ist, dass die Connectoren auch genutzt werden können, um nur Daten von Plattformen in das SIOC-Format zu konvertieren ohne sie unbedingt an andere Connectoren auszutauschen. So können diese Daten auch zu anderen Zwecken verwendet werden. Vorstellbar wäre es die Beiträge durch weitere Ontologien mit Zusatzmaterialien wie Folien zu einer Vorlesung zu verbinden und so Beiträge auch für Maschinen verständlich zusammen in einen Kontext zu bringen. Dadurch bekommt man, wenn man nach Fragen zu einen Thema such auch gleichzeitig weiterführende Materialien ausgegeben. Ebenfalls könnten die unterschiedlichen Beiträge mit Fragen zu Inhalten eines Kurses dazu genutzt werden, automatisch Fragenkataloge mit den Fragen generiert werden, bei denen es die meisten Probleme gab. Schlussendlich decken die fünf Plattformen für die hier Connectoren Implementiert wurden nicht alle Orte ab, die für Diskussionen genutzt werden und voneinander isoliert sind. Es gibt noch eine Menge weiterer die von einer Verwendung des SIOC-Formats profitieren könnten.

% section ausblick (end)

% chapter abschlussbetrachtung (end)