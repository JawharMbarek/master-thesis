%!TEX root = ../thesis.tex

\chapter{Zusammenfassung und Ausblick} % (fold)
\label{cha:zusammenfassung_und_ausblick}

Im letzten Kapitel sollen noch einmal alle Ergebnisse diese Arbeit zusammengefasst und einen Ausblick auf weiterführende Themen gegeben werden. In der Einleitung wurde beschrieben, dass Diskussionen ein wichtiger Bestandteil des E-Learnings ist, aber diese oftmals auf verschiedene Plattformen verteilt stattfinden. Dadurch verpassen Personen, die nur auf einer dieser Plattformen präsent sind, vielleicht für sie wichtige Informationen auf einer anderen. Außerdem werden die selben Diskussionsthemen immer wieder an unterschiedlichen Orten an den unterschiedlichsten Orten doppelt und dreifach behandelt. Aus diesem Grund sollte ein System entwickelt werden das den Austausch von Diskussionen zwischen den unterschiedlichen Plattformen ermöglicht. 

In Kapitel \ref{cha:analyse} wurde analysiert, welche Schritte für eine solche Synchronisation nötig sind. Zuerst musste eine Zwischenformat gefunden werden in das die Daten der unterschiedlichen Plattformen konvertiert werden konnten, da dieser Weg effizienter ist, als die einzelnen Formate untereinander zu konvertieren. Also ein solches Zwischenformat bat sich die SIOC Ontologie in Verbindung mit FOAF wunderbar an. Ontologien bietet nicht nur einen guten Ansatz für die Integration von unterschiedlichen Datenformaten, mit RDF ist es sogar möglich dass Programme diese Daten verstehen und aus ihnen neues Wissen ableiten können. Um die Daten letztendlich in das Zwischenformat konvertieren und verarbeiten zu können, braucht es eine einheitliche Schnittstelle mit der dies umgesetzt werden kann. Zusätzlich wurde noch ein System gebraucht, dass die konvertierten Daten zwischen den Schnittstellen für die einzelnen Plattformen austauscht. Zu eignete sich der Austausch über Nachrichten am besten, da dadurch die einzelnen Schnittstellen zeitlich und räumlich entkoppelt werden konnten und nicht voneinander abhängig sind. Als Basis für dieses Nachrichtensystem wurde Camel ausgewählt. Mit dieser Java-Bibliothek können die Routen, welche die Nachrichten nehmen, flexibel konfiguriert werden und sind so leicht erweiterbar. Privatsphäre spielt in der heutigen Zeit ebenfalls ein wichtige Rolle. Deswegen muss es Benutzern es ermöglicht werden das automatischen Lesen und/oder Schreiben für seine Beiträge zuzustimmen oder abzulehnen.

Das Kapitel \ref{cha:eigener_ansatz_social_online_community_connectors_socc_} widmet sich der Beschreibung des Systems, welches die Anforderungen aus Kapiel \ref{cha:analyse} erfüllen soll. Dieses SOCC genannte System definiert dazu Conectoren, welche die Schnittstellen zu den einzelnen Plattformen abstrahiert und die unterschiedlichen Formate in das SIOC-Format konvertiert. Ein Connector bestehen aus drei Komponenten die je für eine bestimmte Teilaufgabe des Systems verantwortlich sind. Der StructureReader hilft dabei die Struktur der jeweiligen Plattform zu lesen und  in SIOC abzubilden, so dass die Beiträge von den richtigen Stellen gelesen und an die richtigen Stellen geschrieben werden können. Der PostReader ist dabei für das Lesen von Beiträgen und deren Konvertierung vom Format der gelesenen Plattform in das SIOC-Format. Für dem umgekehrten Weg wird dann der PostWriter eingesetzt. Für den Betrieb der Connectoren brauchen diese einige Informationen von den Benutzern, für den Zugriff auf die einzelnen APIs. Für diese Informationen konnte auf der vorhandene Service Modul von SIOC für API Beschreibungen, die Basic Access Control Ontologie für die Autorisierung, sowie auf die aus vorhandenen Ontologien zusammengesetzte und erweiterte Services Authentication Ontologie zur Authentifizierung aufgebaut werden. Gespeichert wird alles in einen RDF-Triplestore und erlaubt so einen einfachen Zugriff auf die untereinander verbunden Datensätze. Für die Integration der Connectoren in Camel wurde die Komponente SOCC-Camel entwickelt. Mit den EIP aus Camel können so die Connectoren auf verschiedenste Weisen verbunden werden.

Im letzten Kapitel 


\section{Ausblick} % (fold)
\label{sec:ausblick}

% section ausblick (end)

% chapter abschlussbetrachtung (end)