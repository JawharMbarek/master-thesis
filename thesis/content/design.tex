%!TEX root = ../thesis.tex

\chapter{Design}
\label{ch:design}

\begin{itemize}
    \item Eie sollen die Anforderungen aus Kapitel \ref{ch:requirementanalysis} letztendlich in ein funktionstüchtiges Programm umgesetzt werden. 
\end{itemize}

%------------------------------------------------------------------------------

\section{Wahl des Zwischenformats} % (fold)
\label{sec:selection_of_the_itermediate_format}
\begin{itemize}
    \item Warum kein eigenes Format
    \item Warum nicht RSS 
    \item Warum nicht Dublincore + X
    \item Warum SIOC + FOAF + X
\end{itemize}

\subsection{RSS} % (fold)
\label{sub:rss}

\subsection{Dublin Core} % (fold)
\label{sub:dublin_core}

\subsection{SIOC und FOAF} % (fold)
\label{sub:sioc_und_foaf}

% subsection sioc_und_foaf (end)

%------------------------------------------------------------------------------

\section{Social Online Community Connectors} % (fold)
\label{sec:connectors}

Einleitung über den Sinn und Zweck eines Connectors in SOCC

\subsection{Aufbau} % (fold)
  \label{sub:aufbau}
  
\subsection{SIOC Service Auth Ontology} % (fold)
\label{sub:sioc_service_auth_ontology}

\subsection{SOCC ConnectorCFG Ontology} % (fold)
\label{sub:socc_connectorcfg_ontology}

\subsection{W3 Web ACL Ontology} % (fold)
\label{sub:w3_web_acl_ontology}


%------------------------------------------------------------------------------

\section{Routen der Einträge} % (fold)
\label{sec:messageing}

\begin{itemize}
    \item Eigener Ansatz mit JMS
    \item Apache Camel
\end{itemize}

%------------------------------------------------------------------------------
