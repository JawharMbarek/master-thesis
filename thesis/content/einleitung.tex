%!TEX root = ../thesis.tex

\chapter{Einleitung} % (fold)
\label{cha:einleitung}

\begin{itemize}
    \item Motivation/Problemstellunng
    \item Anforderungen: was soll entwickelt werden
    \item Übersicht über alle Kapitel
\end{itemize}

% Web 2.0
% User generated content
% mehr Wissen als jemand verstehen kann
% sehr verteilt
% wenig verlinkt, vieles mehrfach vorhanden
% Austausch ist wichtig zum lernen -> soziale Kompetenzen
% Lerngruppe oft auf verschiedene sozialen online Netzwerken (SON) unterwegs
% SON oft "Daten-Silos"
% Facebook als LMS
%  * + fb weit verbraitet
%  * - Nur Bilder und Videos unterstützt (rest auf GDocs + Links)
%  * "the tutor noticed that it was quite troublesome to add teaching materials" [Using the Facebook group as a LMS]


\todo[inline]{Use cases, Vision und Anforderungen. Auf die kann dann in der Analyse genauer drauf eingegangen werden}
\todo[inline]{Anfang}
Jedem ist es heutzutage möglich eigene Inhalte ohne großen Aufwand ins Internet zu stellen und anderen an seinen Wissen teilhaben zu lassen. Dadurch was es noch nie so einfach an Wissen Dritter zu kommen wie heute und dieses Wissen in seinen eigenen Lernprozess mit einfließen zu lassen. 

\medskip

Soziale Netzwerke erleben seit den letzten Jahren einen großen Boom. Sie ermöglichen es mit seinen Freunden in Kontakt zu bleiben auch über zeitliche und räumliche Hürden hinweg. Diese sozialen Netzwerke erlauben es auch ohne physische Präsenz sich untereinander zu organisieren. Qiyun Wang et. al. \cite{Wang2012} zeigten, dass zum Beispiel Facebook\footnote{\url{http://www.facebook.com}} sich hervorragend für den Einsatz als Lernplattform beziehungsweise Learning Management System (LMS) eignet. Gerade die Organisation der Lerngruppe als auch die Benachrichtigung über Ereignisse funktionierte reibungslos. Jedoch uneingeschränkt konnte Facebook als LMS nicht empfohlen werden. Bemängelt wurden unter anderem die aufwändige Integration von Lernmaterialien \enquote{tutor noticed that it was quite troublesome to add teaching materials}\cite[S.\,435]{Wang2012}. Aber auch da es nicht möglich war Diskussionen in einzelne Themen zu unterteilen sondern alle Beiträge nur chronologisch angeordnet sind wurde bemängelt.

\medskip

Für die meisten Diskussionen, welche im Internet geführt werden, sind Foren eine beliebte Plattform. Hier können auf einfachen Weg neue Fragen gestellt, bestehende Fragen erweitert oder beantwortet werden. Da für das Schreiben in Forum zum Großteil Pseudonyme verwendet werden, ist außerdem eine problemlose Beteiligung an Diskussionen von zum Beispiel Studenten möglich die sich in einer Vorlesung nicht trauen Fragen zu stellen. Ebenfalls für Personen die sich nicht trauen bestimmte Fragen zu stellen, weil sie diese für zu dumm halten, sind Foren, Blogs oder Wikis ein guter Anlaufpunkt. In diesen Fall kann eine oftmals vorhandene Suchfunktion benutzt werden,  um nach dieser oder ähnlichen Fragen zu suche die andere zuvor gestellt beziehungsweise beantwortet habe.

\medskip

%Daten-Silos
%Keine direkte Verlinkung

In der Regel gibt es für ein Thema nicht nur eine sondern einen große Anzahl verschiedener Orte die ein bestimmtes Thema behandeln. Im Fachbereich Informatik der Technischen Universität Darmstadt existiert zum Beispiel von der Fachschaft betriebenes Forum mit Unterforen zu den verschiedensten Veranstaltungen. Gleichzeitig wird nebenbei das LMS Moodle\footnote{\url{https://moodle.org/}} eingesetzt, in dem zu einigen Vorlesungen eigne Foren betrieben werden. Hierbei kann es passiert, dass einzelne Diskussionen nur auf einer der beiden Plattformen geführt werden und möglicherweise wichtige Beiträge verpasst werden. Im umgedrehten Fall werden Diskussionen mehrfach an verschiedenen Orten geführt, dadurch entsteht eine Redundanz der gleichen Informationen. Will man hierbei zur Vermeidung auf weitere Quellen (Beiträge, Vortragsfolienfolien, $ \dots $) verweisen, bleibt im Normalfall nur die Möglichkeit dies schriftlich in der Form \enquote{schau bei Veranstaltung x in Foliensatz y auf Folie z} zu verweisen. Eine richtige Verknüpfung findet hier in den seltensten Fällen statt. 


\todo[inline]{Use cases, Vision und Anforderungen. Auf die kann dann in der Analyse genauer drauf eingegangen werden}



% chapter einleitung (end)