%!TEX root = ../thesis.tex

\chapter{Einleitung} % (fold)
\label{cha:einleitung}

\begin{itemize}
    \item Motivation/Problemstellunng
    \item Anforderungen: was soll entwickelt werden
    \item Übersicht über alle Kapitel
\end{itemize}

% Web 2.0
% User generated content
% mehr Wissen als jemand verstehen kann
% sehr verteilt
% wenig verlinkt, vieles mehrfach vorhanden
% Austausch ist wichtig zum lernen -> soziale Kompetenzen
% Lerngruppe oft auf verschiedene sozialen online Netzwerken (SON) unterwegs
% SON oft "Daten-Silos"
% Facebook als LMS
%  * + fb weit verbraitet
%  * - Nur Bilder und Videos unterstützt (rest auf GDocs + Links)
%  * "the tutor noticed that it was quite troublesome to add teaching materials" [Using the Facebook group as a LMS]

\todo[inline]{Anfang}
Jedem ist es heutzutage möglich eigene Inhalte ohne großen Aufwand ins Internet zu stellen und anderen an seinen Wissen teilhaben zu lassen. Dadurch was es noch nie so einfach an Wissen Dritter zu kommen wie heute und dieses Wissen in seinen eigenen Lernprozess mit einfließen zu lassen. 

Soziale Netzwerke erleben seit den letzten Jahren einen großen Boom. Sie ermöglichen mit seinen Freunden in Kontakt zu bleiben über zeitliche und räumliche Hürden hinweg. Diese sozialen Netzwerke erlauben es auch ohne physische Präsenz sich untereinander zu organisieren. Qiyun Wang et. al. \cite{Wang2012} zeigten, dass zum Beispiel Facebook\footnote{\url{http://www.facebook.com}} sich hervorragend für den Einsatz als Lernplattform beziehungsweise Learning Management System (LMS) eignet. Gerade die Organisation der Lerngruppe als auch die Benachrichtigung über Ereignisse funktionierte reibungslos. Jedoch uneingeschränkt konnte Facebook als LMS nicht empfohlen werden. Bemängelt wurden unter anderem die aufwändige Integration von Lernmaterialien \enquote{tutor noticed that it was quite troublesome to add teaching materials}\cite[S.\,435]{Wang2012}. Aber auch da es nicht möglich war Diskussionen in einzelne Themen zu unterteilen sondern alle Beiträge nur chronologisch angeordnet sind ist negativ aufgefallen. 





% chapter einleitung (end)