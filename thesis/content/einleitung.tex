%!TEX root = ../thesis.tex

\chapter{Einleitung} % (fold)
\label{cha:einleitung}

% Web 2.0
% User generated content
% mehr Wissen als jemand verstehen kann
% sehr verteilt
% wenig verlinkt, vieles mehrfach vorhanden
% Austausch ist wichtig zum lernen -> soziale Kompetenzen
% Lerngruppe oft auf verschiedene sozialen online Netzwerken (SON) unterwegs
% SON oft "Daten-Silos"
% Facebook als LMS
%  * + fb weit verbraitet
%  * - Nur Bilder und Videos unterstützt (rest auf GDocs + Links)
%  * "the tutor noticed that it was quite troublesome to add teaching materials" [Using the Facebook group as a LMS]


%\todo[inline]{Den Einstieg könnten man auch unter das Thema Web 2.0 und E-Learning 2.0 stellen}


Durch die Omnipräsenz des Internets im heutigen Alltag haben sich viele Bereiche unseres Lebens sehr verändert. Unter anderem die Art wie wir uns weiterbilden und neue Dinge lernen verlagert sich immer mehr dort hin. Prägend für diesen Umbruch ist der Begriff des \enquote{E-Learnings}. Besonders neue Technologien im Zuge des so genanten Web 2.0 wie Blogs, Wikis Diskussionsseiten und sozialer Netzwerke machen es immer leichter neues Wissen zu erwerben und es mit anderen zu teilen. Gerade beim Lernen spielt die Diskussion mit Gleichgesinnten eine wichtige Rolle \cite{Downes2005}. Es wurden Studien durchgeführt, die zeigen dass Studenten welche sich an online Diskussionen teilnehmen dazu tendieren bessere Noten zu bekommen, als solche die nicht teilnahmen \cite{Davies2005,BJET:BJET780}. Auch für zurückhaltende Studenten ist E-Learning eine Verbesserung, da sie sich so eher an Diskussion beteiligen als zum Beispiel in der Vorlesung oder der Lerngruppe \cite{Huang:2009:EPF:1516241.1516267}. 

Qiyun Wang et. al. \cite{Wang2012} zeigen in ihrer Studie, dass sich Gruppen im sozialen Netzwerk Facebook für den E-Learning Einsatz als Learning Management System (LMS) gut nutzen lassen. Teilnehmer konnte auf der Gruppenseite durch Kommentare und Chats mit einander diskutieren. Gerade die Organisation der Lerngruppe als auch die Benachrichtigung über Ereignisse funktionierte reibungslos. Jedoch uneingeschränkt konnte Facebook als LMS nicht empfohlen werden. Bemängelt wurde unter anderem die aufwändige Integration von Lernmaterialien \enquote{tutor noticed that it was quite troublesome to add teaching materials}\cite[S.\,435]{Wang2012} und dass es nicht möglich war Diskussionen in einzelne Themen zu unterteilen. Alle Kommentare wurden nur als eine chronologische Liste dargestellt. Seit 2013 ist es aber auch auf Facebook möglich auf Kommentare direkt zu antworten\footnote{\url{https://www.facebook.com/notes/facebook-journalists/improving-conversations-on-facebook-with-replies/578890718789613}} und so sind auch Foren-ähnliche Diskussionen realisierbar. Jedoch ist diese Erweiterung auf \enquote{Pages} beschränkt. Über eine Ausweitung auf Gruppenseiten ist nichts bekannt. 

Aber nicht nur soziale Netzwerke sind für Diskussionen innerhalb von E-Learning geeignet, Foren oder Blogs sind ebenfalls sehr beliebte Plattformen.Jedoch ein Problem bei der Nutzung des Internets zum Lernen liegt darin, dass es in der Regel nicht nur eine Plattform genutzt wird, sondern häufig mehrere simultan. Zum Beispiel könnte für einen Kurs ein eigenes Forum im LMS des Veranstalters und nebenbei noch eine Gruppe in Facebook existieren. Teilnehmer, die vorzugsweise nur eine eine der Plattformen nutzen, erhalten vielleicht von wichtigen Diskussion auf der anderen Plattform keine Kenntnis. Durch diese Inselbildung werden Themen mehrfach behandelt, da Suchfunktionen nur innerhalb der eigenen Plattform suchen und von der Existenz in der Anderen nichts wissen. Eine Integration von zusätzlichen Wissensquellen ist nur schwer möglich und erfolgt immer wieder nur in verbaler Form wie \enquote{Schau dir auf der Seite x den Artikel y an}.

Aus diesen Gründen soll in dieser Arbeit ein Ansatz entwickelt werden der er es ermöglicht verteilte Diskussionen zusammen zu führen und wiederverwenden zu können. Ein solcher Ansatz muss dazu mehrere Anforderungen erfüllen. Da es sich bei online Plattformen in der Regel um abgeschottete \enquote{Datensilos}\cite{Berners-Lee2011} handelt auf die nur über zum Großteil heterogene Schnittstellen zugegriffen werden kann, ist es hier wichtig eine einheitlich Schnittstelle für den Zugriff auf die gespeicherten Diskussionsdaten zu schaffen. Nicht nur in der Art des Zugriffs unterscheiden sich die einzelnen Plattformen, auch das Format der Daten ist davon Betroffen. Um Diskussionen zwischen den Plattformen überhaupt austauschen zu können ist demzufolge eine Umwandlung in ein gemeinsames Datenformat notwendig, welches erst eine Interoperabilität möglich macht. Als letztes muss noch ein System zur automatischen Synchronisation entwickelt werden wodurch verteilte Diskussionen aktuell gehalten werden können.

Diese Arbeit gliedert sich dazu in folgende Kapitel:
\todo[inline]{Kapitelbeschreibung}

%\todo[inline]{Use cases, Vision und Anforderungen. Auf die kann dann in der Analyse genauer drauf eingegangen werden}

% chapter einleitung (end)