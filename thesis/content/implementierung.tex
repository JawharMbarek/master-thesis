%!TEX root = ../thesis.tex

\chapter{Implementierung} % (fold)
\label{cha:implementierung}

\section{Implementierung der Connectoren} % (fold)
\label{sec:implementierung_der_connectoren}

Das muss rein:
\begin{itemize}
    \item Mapping nach SIOC
    \item Zugriff über die API
    \item Probleme bei der Implementierung
\end{itemize}

\subsection{Moodle} % (fold)
\label{sub:moodle_connector}

\begin{itemize}
    \item Eingebaute REST Schnittstelle, aber kein Lesen von Beiträgen
    \item WebService Plugin MoodleWS (REST oder SOAP)
    \begin{itemize}
        \item https://github.com/patrickpollet/moodlews
        \item ClientAPI existieren von selber Autor
        \item REST defekt, kein schreiben von Beiträgen möglich
        \item SOAP funktioniert mehr oder weniger
        \item Verschluckt Fehlermeldungen
        \item kein lesen einzelner Posts/Threads/Foren
        \item SOAP ClientAPI neu generieren, weil vorhandene nicht mit 2.4 funktioniert.
        \item Username/Password + Session Token/Id
        \item “Use an auto generated wsdl” -> No
        \item schreiben von neuen Beitrag direkt in thread nur als Antwort auf ersten Beitrag möglich
        \item Rückgabe aller Beiträge in einem Objekt
    \end{itemize}
\end{itemize}

\subsubsection{SIOC Mapping} % (fold)
\label{ssub:moodle_sioc_mapping}

\subsubsection{API} % (fold)
\label{ssub:moolde_api}

\subsubsection{Herausforderungen} % (fold)
\label{ssub:moodle_herausforderungen}

% subsubsection moodle_herausforderungen (end)

% subsubsection moodle_api (end)

% subsubsection moodle_sioc_mapping (end)

% subsection moodle_connector (end)

\subsection{Facebook} % (fold)
\label{sub:facebook_connector}

\begin{itemize}
    \item REST API + JSON
    \item keine offizielle Java API für Desktop -> Web + Mobile only
    \item GraphAPI, Facebook Query Language
    \item OAuth 2.x
    \begin{itemize}
        \item kein Refreshtoke
        \item Token Haltbarkeit 2h (2 Monate, wen extended)
        \item token nur über webbrowser
    \end{itemize}
    \item RestFB alternative Java API für die REST Schnittstelle der GraphAPI
    \item Typ der zurückgelieferten Daten nicht anhand der URI erkennbar, häufig erst durch Angabe von \emph{metadata=1}
    \item beim herunterladen einzelner Posts nicht immer erkennbar wo sie geschrieben wurden
\end{itemize}

\subsubsection{SIOC Mapping} % (fold)
\label{ssub:facebook_sioc_mapping}

\subsubsection{API} % (fold)
\label{ssub:facebook_api}

\subsubsection{Herausforderungen} % (fold)
\label{ssub:facebook_herausforderungen}

% subsubsection facebook_herausforderungen (end)

% subsubsection facebook_api (end)

% subsubsection facebook_sioc_mapping (end)

% subsection facebook_connector (end)

\subsection{Google+} % (fold)
\label{sub:google_plus_connector}

\begin{itemize}
    \item Einfach REST API + JSON
    \item OAuth
    \begin{itemize}
        \item Refreshtoken (token laufen quasi nie ab)
        \item holen von token ohne webbrowser möglich
    \end{itemize}
    \item Objekte aufgebaut aus Actor (wer machte was), Verb(wie machte er es), Object (wtas machte er) + Metadata
    \item verschieden Sprachen + Plattformen
    \item lesen nur von öffentlichen Beiträgen
    \item kein Schreiben von Beiträgen
\end{itemize}

\subsubsection{SIOC Mapping} % (fold)
\label{ssub:google_plus_sioc_mapping}

\subsubsection{API} % (fold)
\label{ssub:google_plus_api}

\subsubsection{Herausforderungen} % (fold)
\label{ssub:google_plus_herausforderungen}

% subsubsection google_plus_herausforderungen (end)

% subsubsection google_plus_api (end)

% subsubsection google_plus_sioc_mapping (end)

% subsection google_plus_connector (end)

\subsection{Youtube } % (fold)
\label{sub:youtube_connector}

\begin{itemize}
    \item Aktueller Umbau der API (ähnlich google+) v3
    \begin{itemize}
        \item keine lesen von kommentaren
        \item kein schreiben
    \end{itemize}
    \item alte GData Feed API v2 basiert auf RSS + Youtube Erweiterung
    \item Mapping teilweise durch basis auf RSS einfach, manchmal auch nicht
    \item Wichtigen Metadaten nur implizit vorhanden (comment id in uri aber nicht in datenformat)
\end{itemize}

\subsubsection{SIOC Mapping} % (fold)
\label{ssub:youtube_sioc_mapping}

\subsubsection{API} % (fold)
\label{ssub:youtube_api}

\subsubsection{Herausforderungen} % (fold)
\label{ssub:youtube_herausforderungen}

% subsubsection youtube_herausforderungen (end)

% subsubsection youtube_api (end)

% subsubsection youtube_sioc_mapping (end)

% subsection youtube_connector (end)

\subsection{Canvas} % (fold)
\label{sub:canvas_connector}

\begin{itemize}
    \item relativ neues LMS
    \item super Bedienung
    \item super REST API
    \item keine Java API
    \item rudimentäre Eigenentwicklung einer Java API, Funktionsweise ähnlich  G+
    \item viel API Funktionen wohl nicht extern nutzbar (UserProfil lesen, vll. Falsche Berechtigung -> test nötig)
\end{itemize}

\subsubsection{SIOC Mapping} % (fold)
\label{ssub:canvas_sioc_mapping}

\subsubsection{API} % (fold)
\label{ssub:canvas_api}

\subsubsection{Herausforderungen} % (fold)
\label{ssub:canvas_herausforderungen}

% subsubsection canvas_herausforderungen (end)

% subsubsection canvas_api (end)

% subsubsection canvas_sioc_mapping (end)

% subsection canvas_connector (end)

% section implementierung_einiger_connectoren (end)

% chapter implementierung (end)