%!TEX root = ../thesis.tex

\chapter{Anforderungsanalyse}
\label{ch:requirementanalysis}

\enquote{<< hier fehlt ein Anfang >>}

Soziale Netzwerk A speichert die Daten der Nachricht in sein eigens Format A. Um diese Nachrichten  in das soziale Netzwerk B transferieren zu können, müssen zuerst die Daten über eine API von den Servern des sozialen Netzwerks A herunter laden. Da in der Regel nicht automatisch bekannt ist, wann eine neue Nachricht vorhanden ist, müssen die Server in zeitlichen Abständen abgefragt werden und die zurückgelieferten Daten nach neuen Nachrichten durchsucht werden (Polling). Wurden neue Nachrichten gefunden können diese nicht direkt an das soziale Netzwerk B geschickt werden, da sich diese in der Regel im verwendeten Format unterschieden. Die einfachste Möglichkeit wäre nun die Daten von Format A nach Format B zu konvertieren. Bei zwei Netzen ist dies noch sehr einfach. Es müsste lediglich nur ein Konverter von A nach B und einer von B nach A implementiert werden. Kämme nun ein drittes Netzwerk C hinzu, wären sechs Konverter nötig (A $\Rightarrow$ B, A $\Rightarrow$ C, B $\Rightarrow$ A, B $\Rightarrow$ C, C $\Rightarrow$ A und C $\Rightarrow$ B). Nimmt man nun n als eine beliebige Anzahl Netzwerke, entspricht die Anzahl der Konverter $k= n*(n-1)$, da für jedes Netzwerk ein Konverter in alle anderen Netzwerke nötig wird. Aber auch die Konverter selber sind eine Herausforderung für sich. Nicht alle Formate sind gleich aufgebaut. Im einfachsten Fall unterscheiden sich die einzelnen Daten, aber ist es auch möglich das ein Netzwerk die Features eines anderen nicht oder in einer anderen Form anbietet. Zum Beispiel das Bewerten von einzelnen Nachrichten. Das Eine bietet Bewertungen im fünf Sterne System an, das Andere nur positive und negative Bewertungen. 

Um beide Probleme zu beheben wäre die Einführung eines Zwischenformats, iIm Folgenden Format X genannt, eine der elegantesten Möglichkeiten. Würden alle Konverter erst in dieses Format X und von diesen in das Zielformat transformieren wären für jedes weitere Netzwerk nur zwei neue Konverter von Nöten. Für $n=2$ wäre dies zwar schlechter, da vier statt zwei und im Fall $n=3$würden ebenfalls sechs Konverter erzeugt werden müssen, doch schon bei vier Netzwerken würde die Anzahl von zwölf auf acht sinken. Für jedes $n>3$ ist der Weg über ein Zwischenformat vom Arbeitsaufwand effizienter. 
