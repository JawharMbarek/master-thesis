%!TEX root = ../thesis.tex
{ %
\renewcommand\abstractname{Kurzfassung}
\begin{abstract}[1]
In der heutigen Zeit verlagert sich unser Leben immer mehr in das World Wide Web. Nicht nur Kommunikation zwischen Freunden, aber auch Diskussionen um Lerninhalte werden auf unterschiedlichsten Plattformen im Web geführt. Gerade Lernplattformen wie Moodle oder soziale Online-Netzwerke sind dazu sehr beliebt geworden. Diese Diskussionen finden allerdings isoliert voneinander statt und Informationen gelangen selten über die Grenzen einer Plattform hinaus. So sind diese mehrfach über das gesamte Web verteilt oder nur an wenigen, versteckten Orten zu finden. Ein Austausch der Daten ist aber durch die unterschiedlichen Datenformate äußerst schwierig.

Aus diesem Grund soll mit dieser Masterarbeit ein System entwickelt werden, dass Beiträge aus verschieden Diskussionsquellen untereinander synchronisieren kann. Dazu wird analysiert welche Schritte dazu gemacht werden müssen und welche Komponenten dazu nötig sind. Im Anschluss an diese Analyse wird der hier entwickelte Ansatz der \emph{Social Online Community Connectors} vorgestellt. Auf der Basis des \emph{Resource Description Frameworks} und der Ontologie für \emph{Semantically-Interlinked Online Communities} ist es mit diesem Ansatz möglich, Diskussionen zwischen verschiedenen Plattformen und Datenformaten auszutauschen. Eine Implementierung erfolgt am Ende beispielhaft für die Plattformen Moodle, Canvas, Facebook, Google+ und Youtube.
\end{abstract}
}

{% speak english from here
\selectlanguage{english}
\begin{abstract}[2]

Today, our lives shifts more and more into the World Wide Web. Not only communication between friends, but also discussions about learning content are made in different platforms inside the web. The most popular platforms for these type of discussions are learning management systems like Moodle or online social networks similar to Facebook. However, these discussions are done isolated from each other and the informations are rarely transfered between them. So these information are distributed all over the web multiple times or can only be found on few, different places. Because of the different data formats, the exchange of this information is difficult.

For this reason, a system should be developed that can synchronize discussions of different platforms. A following analysis shows what steps must be taken and what components are needed to achieve this goal. According to this analysis, the developed \emph{Social Online Community Connectors} approach is presented. Based on the \emph{Resource Description Frameworks} and the \emph{Semantically-Interlinked Online Communities} ontology it's possible to exchange discussions between different platforms and data formats. An exemplary implementation of this approach will be done for the platforms Moodle, Canvas, Facebook, Google+ and Youtube.

\end{abstract}
} % Ab hier wieder in deutsch