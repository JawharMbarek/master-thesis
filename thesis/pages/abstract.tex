%!TEX root = ../thesis.tex
{ %
\renewcommand\abstractname{Kurzfassung}
\begin{abstract}[1]
In der heutigen Zeit verlagert sich unser Leben immer mehr in das World Wide Web. Nicht nur Kommunikation zwischen Freunden, aber auch Diskussionen um Lerninhalte werden auf unterschiedlichsten Plattformen im Web geführt. Gerade Lernplattformen wie Moodle oder soziale Online-Netzwerke sind dazu sehr beliebt geworden. Diese Diskussionen finden allerdings isoliert voneinander statt und Informationen gelangen selten über die Grenzen einer Plattform hinaus. So sind diese mehrfach über das gesamte Web verteilt und nur an wenigen, bestimmten Orten zu finden. Ein Austausch der Daten ist aber durch die unterschiedlichen Datenformate äußerst schwierig.

Aus diesem Grund soll mit dieser Masterarbeit ein System entwickelt werden, dass Beiträge aus verschieden Diskussionsquellen untereinander synchronisieren kann. Dazu wird analysiert welche Schritte dazu gemacht werden müssen und welche Komponenten dazu nötig sind. Im Anschluss an diese Analyse wird der hier entwickelte Ansatz der Social Online Community Connectors vorgestellt. Auf Basis des Resource Description Frameworks und der Ontologie für Semantically-Interlinked Online Communities ist es mit diesen Ansatz möglich, Diskussionen zwischen verschiedenen Plattformen auszutauschen. Eine Implementierung erfolgt am Ende beispielhaft für die Plattformen Moodle, Canvas, Facebook, Google+ und Youtube.
\end{abstract}
}

{% speak english from here
\selectlanguage{english}
\begin{abstract}[2]
\todo[inline]{Content of the abstract}
\end{abstract}
} % Ab hier wieder in deutsch